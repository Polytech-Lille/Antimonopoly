\usepackage{hyperref}
\author{
        Bragagnolo, Santiago
}
\title{TP GIS4 - Anti-monopoly 2 }
\date{\today}
 
\documentclass[12pt]{article}
 
\begin{document}
\maketitle
Cet exercise est base sur un exercise propose par le cour de Paradigmes de programation, au UTN, Argentine .

\section{Introduction}
Vu le grand succès de la première version de Antimonopoly, il a été décidé de compliquer la
règles du jeu afin de le rendre encore plus divertissant. En particulier,
ajouts suivants:


\section{Des nouveaux investissements}
Dans la deuxième version du jeu, des investissements dans l'immobilier et les services de la communication sont ajoutés.




\begin{itemize}
    \item Services de communication. La première fois que quelqu'un tombe dans l'un de ces
Casilleros paie le loyer correspondant comme tout investissement. La deuxième fois
ce joueur tombe dans le casier paie le double de ce qui est indiqué, la troisième fois triple
et ainsi de suite.

    \item L'immobilier. Comme tous les investissements, ils ont une valeur d'achat et peuvent être
un joueur ou être entre les mains de l'Etat. Ils représentent des terrains sur le
qui peuvent être construits et ainsi augmenter leur valeur et les revenus qu'ils produisent.
    \begin{itemize}
        \item Tant que rien d'autre n'est construit sur eux, ils sont comme des investissements
commun.
        \item Le joueur qui détient l'investissement, lorsqu'il tombe dans la boîte de son
investissement, vous pouvez choisir de construire et donc d'augmenter son investissement.
        \item Par conséquent, le joueur qui tombe dans un de ces investissements, s'il y a
constructions supplémentaires, vous devez payer un loyer plus élevé.
        \item Les possibilités de construction sont:
            \begin{itemize}
                \item Département: a une valeur équivalente à la moitié de la valeur du
                terrain. Il génère un revenu de 10\% de la valeur totale de l'investissement.
                Entre 1 et 4 appartements peuvent être construits sur un seul terrain.
                \item Bâtiment: il a une valeur égale à celle du terrain. Produit un revenu de
                40\% de la valeur totale de l'investissement. Vous pouvez créer un seul
                construire sur un terrain.
            \end{itemize}
        \item S'il a plusieurs constructions sur un terrain, la valeur de l'investissement
et le montant du benefit est calculé comme la somme de tous.
        \item Les décisions concernant ces investissements dépendent du style de jeu
chaque joueur:
            \begin{itemize}
                \item Prudent: les mêmes conditions que pour tout investissement, mais
Il construit également des appartements d'abord, un à la fois, et
lorsque vous avez les quatre départements, vous pouvez investir dans la construction
un édifice.
                \item Agressif: construisez tout ce que vous pouvez en une seule fois.
            \end{itemize}

\end{itemize}






\section{Style de joueur}

    Le style de jeu d'un joueur peut varier tout au long du jeu et peut être intégré
de nouveaux styles.

\section{Cartes Fortune}

Cartes Fortune
Dans le jeu, il y a une pile de cartes, indiquant les situations qui affectent
en quelque sorte à un joueur (et non à un autre joueur).
 Quand un joueur tombe dans un casier "Fortune", il doit prendre la première carte de la pile et faites ce
  qu'on vous dit de faire. 
  Mettez ensuite la carte sous la pile. 
  Par exemple, ils peuvent indiquer que le joueur avance d'un nombre de casiers, 
  qu'il doit paier ou gagner de l'argent ou qu'il doit éliminér un investissement.


Requerimientos
    Definir un tablero incluyendo los nuevos casilleros.
    Realizar las mismas funcionalidades de la primera versión del Monopoly, teniendo
en cuenta las nuevas indicaciones.
    Definir un nuevo estilo de jugador inventando sus características. Hacer que un
determinado momento algunos jugadores cambien su estilo de juego por otro y
que el juego continúe.
    Probar también alguna de las variantes indicadas en la primera parte, con las
nuevas indicaciones.

\section{Exigences du projet}
     Définissez un plateau de jeu incluant les nouvelles cases.

     Effectuer les mêmes fonctionnalités de la première version de AntiMonopoly, en prenant
    en compte les nouvelles indications:
    \paragraph{ Définissez un nouveau style de joueur} en inventant ses caractéristiques. 
    \paragraph{ Faites que au moment donné}, certains joueurs changent de style de jeu pour un autre et
laissez le jeu continuer.
    \paragraph {Essayez} également certaines des variantes indiquées dans la première partie, avec le
nouvelles indications.


\section{Bonus} 
    \subsection{Dynamique du jeu}
Développez automatiquement toutes les dynamiques de jeu, de sorte qu'avec envoyer un message initial, 
tout le jeu continue jusqu'à ce qu'il y ait un seul gagnant. 
    Pour cela, entre autres choses, le dé doit être modélisé de manière à comportement aléatoire. 
    Il est recommandé de vous informer d'une manière ou d'une autre sur les principaux événements du jeu ou suspendre l'exécution.


\end{document}

