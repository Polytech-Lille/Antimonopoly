\usepackage{hyperref}
\author{
        Bragagnolo, Santiago
}
\title{TP GIS4 - Anti-monopoly}
\date{\today}
 
\documentclass[12pt]{article}
 
\begin{document}
\maketitle
Cet exercise est base sur un exercise propose par le cour de Paradigmes de programation, au UTN, Argentine .

\section{Introduction}

Vous souhaitez implémenter un jeu de style Monopoly dans lequel les joueurs marchent un chemin où
ils doivent prendre des décisions d'investissement et faire certaines actions, afin d'augmenter votre
patrimoine. Attention à ne pas trop Allez au-dessus, parce que les lois strictes antitrust d'État peut vous obliger à
distribuer votre richesse.


\section{Characteristiques du jeu}

    \subsection{Plateau du jeu}
    La Plateau est constituée d'un chemin circulaire composé d'un certain nombre de casiers,
dans chacun desquels il y a des indications. Il y a une case marquée "sortie" qui
indique le point de départ du jeu.

    \subsection{Des Jouers}
    Il peut y avoir autant de joueurs que souhaité. Chaque joueur a un certain
montant initial de l'argent. Au fur et à mesure que le jeu progresse, chacun de jouers augmente
ou réduis leur argent, selon les décisions d'investissement que chacun prend ou d'autres
circonstances qui vous viennent à l’esprit. Les capitaux propres d'un joueur sont considérés comme équivalents à
l'argent qui a plus l'évaluation de tous les investissements qu'il ait. Si un joueur
il perd tous ses capitaux qu'il ne peut pas continuer à jouer.

    
    \subsection{L'etat}
    L'Etat est un acteur social dont la fonction principale est de réguler l'activité économique des
différents joueurs. Vous avez une certaine quantité de ressources financières.  L'Etat ne participe pas
au jeu en marchant sur le plateau, mais interagissez avec tous les joueurs différentes manières.

    \subsection{La dinamique de jeu}
    Initialement, tous les joueurs sont placés dans la case "sortie". À tour de rôle, chacun
    des joueurs lance un dé et avance d'où il se trouve autant de pas que l'indique le
    dé, arrivant ainsi à un nouveau casier. Là, le joueur doit suivre les instructions
    qui sont indiqués dans la case et prennent les décisions économiques qu'il juge appropriées.
    De cette façon, tous les joueurs avancent à travers le plateau et font leur
    entreprise. Comme l'itinéraire est circulaire, il n'y a pas de point d'arrivée prédéterminé,
    raison pour laquelle ils continuent à tourner jusqu'à ce que l'on souhaite mettre fin au jeu.
    Le jeu est également terminé s'il reste un seul joueur ou si l'État échoue.
    
    \subsection{Casiers}
    Les casiers peuvent être:
    \begin{itemize}
        \item Investissement: si l'investissement est disponible, le joueur décide s'il veut le prendre
    lui ou laissez-la passer. Si l'investissement provient déjà d'un autre joueur, qui tombe dans la boîte
    Vous devez lui payer une somme d'argent basée sur la valeur de l'investissement.
        \item Loi antitrust: oblige le joueur à céder tous ses investissements
    qu'il possède qu'ils dépassent le maximum fixé par l'État. Investissements
    ils reviennent à l'Etat, qui paie le joueur la moitié de leur prix, et ils restent
    à nouveau disponible.
        \item  Bureau Finances publiques : oblige le joueur à payer à l'État un certain pourcentage de son
    capitaux propres, au cas où il dépasserait une certaine valeur minimale. Selon quelle taxe
    traiter, ces valeurs changent (impôt sur le revenu, impôt foncier personnel, etc.)
        \item Subvention: Le joueur reçoit de l'État le montant indiqué dans la case.
        \item Respos: C'est un casier vide dans lequel rien ne se passe. En particulier, la sortie
    en fait partie.
    \end{itemize}

    \subsection{Investissements}
Les investissements sont le principal moyen pour les joueurs de gagner de l'argent. Initialement
Ils sont tous entre les mains de l'Etat et chacun a un prix de référence, qui est le
montant à payer par le joueur lorsqu'il tombe dans la boîte et décide d'assumer ledit
investissement.
Chaque investissement génère un revenu calculé en pourcentage du prix du
référence, qui est facturée au joueur qui tombe dans la boîte. Prix et pourcentages de
les bénéfices d'investissement sont complètement arbitraires.

    \subsection{Style des joueurs}

    Le fait qu'un joueur accepte ou non de conserver un investissement dépend en premier lieu 
    avoir assez d'argent, mais aussi votre style de jeu. 
    Certains styles sont suivant:
        \begin{itemize}
           \item Prudent: vous ne l'acceptez que si vous avez moins d'un certain montant 
           d'investissements et La valeur d'investissement représente moins de 20 \% de vos
           actifs actuels.
           \item  Agressif: Achetez tout.
        \end{itemize}

    Finalement, si un joueur est obligé de se séparer de certains investissements, 
        il choisit lesquels faire en fonction de votre style de jeu:
        \begin{itemize}
            \item Prudent: ceux qui ont les prix les plus élevés.
            \item Agressif: le moins cher.
        \end{itemize}


        
\section{Exigences du projet}
Sur la base de l'approche générale du jeu, le
L'implémentation demandée doit comprendre:

\subsection{Définition initiale}
Créez un plateau de jeu avec plusieurs cases (au moins 10), certains joueurs (au moins 2, un des
chaque style de jeu) et l'État. Tout définir prix, pourcentages, montants d’argent et
valeurs initiales nécessaires. 
Distribuer quantités et valeurs de casier d'une manière que vous considérez comme équilibrée.

\subsection{Simulation}

 Dans la method maine effectuez les étapes suivantes, en choisissant arbitrairement la marque de dés, 
 parmi façon de voir les différentes variantes:
   Localisez tous les joueurs à la sortie.
   Demander à un joueur d'avancer un certain nombre de cases et de faire quoi
 lui correspond.
   Vérifier la situation financière du joueur
   Faites jouer tous les joueurs en terminant un tour.
   Obtenez le classement des actifs des joueurs.
Simulez quelques tours jusqu'à ce qu'il ne reste qu'un joueur.


\subsection{Variantes}
Redémarrez le jeu en inventant les conditions initiales selon différents
"Positions idéologiques", et voyez leur impact sur la simulation du jeu. pour
exemple:
\begin{itemize}
  \item NeoLiberal: Certains joueurs avec beaucoup d'argent et d'autres avec peu, un état avec
mesures antitrust légères.
  \item Socialiste: Tous les joueurs avec le même argent de départ, beaucoup de taxes.
  \item Capitaliste: tous les joueurs agressifs, pourcentages de profit élevés, plus élevés
ratio d'investissement.
  \item Progressiste : plus de joueurs, des lois antitrust strictes
combiné avec des subventions, des pourcentages de profit modérés.
  \item l'Europe apres Covid-19 : (au choix de chacun).
\end{itemize}
\end{document}

